\usepackage[symbols,nopostdot,nonumberlist,section]{glossaries-extra}

\newglossarystyle{supergroup}{%
	\setglossarystyle{super}%
	\renewcommand*{\glsgroupskip}{}%
	\renewcommand{\glossentry}[2]{%
		\tabularnewline%
		\multicolumn{2}{p{\textwidth}}{%
			\raggedright\glsentryitem{##1}\glstarget{##1}{\glossentryname{##1}}%
		}% 
		\vspace{2mm}%
		\tabularnewline%
	}%
	\renewcommand{\subglossentry}[3]{%
		\glssubentryitem{##2}%
		\glstarget{##2}{\glossentryname{##2}}&%
		\raggedright\glossentrydesc{##2}\glspostdescription\space##3\tabularnewline%
	}%
}
\newcommand{\test}[1]{ \def\tst{#1} \ifx\tst\empty \typeout{optional argument was omitted} \else \typeout{optional argument was given: '#1'} \fi}
\newcommand{\glsgroup}[3]{%
	\newglossaryentry{#1}{type=symbols, name={{\large \textbf{#2}} \def\temp{#3}\ifx\temp\empty\else\vspace{2mm}\newline #3\fi}, description={}}
}
\newcommand{\glsent}[4]{\newglossaryentry{{#1:#2}}{sort={#2},type=symbols,name={#3},description={#4},parent={#1}}}
\newcommand{\glsentm}[4]{\glsent{#1}{#2}{\ensuremath{#3}}{#4}}

%\setglossarypreamble[symbols]{Ovaj odjeljak sadrži popis velikog broja oznaka koje se koriste u ovom radu. Za neke skupine oznaka napisana su kratka objašnjenja koja dodatno pojašnjavaju i opravdavaju neke oznake. Pojmovi koje označavaju neke oznake detaljnije su objašnjeni u poglavlju~\ref{chap:osnovni-pojmovi}.}

% Objekti
\glsgroup{o}{Objekti}
{Varijable se označavaju kosim slovima sa serifima, većina konstanti uspravnim slovima sa serifima, a slučajne varijable kosim slovima bez serifa. Vektori se označavaju malim podebljanim slovima, matrice i višedimenzionalni nizovi velikim podebljanim slovima, a skupovi slovima s udvostručenim linijama. }
\glsentm{o}{var}{a,\,A,\,\theta}		{Varijabla (najčešće skalar)}
\glsentm{o}{vec}{\vec a,\,\vec\theta}	{Vektor ili niz (najčešće vektor stupac)}
\glsentm{o}{mat}{\mat A,\,\mat\Theta}	{Matrica ili višedimensionalni niz}
\glsentm{o}{set}{\set A,\,\set\Theta}	{Skup ili multiskup}
\glsentm{o}{const}{\const a,\,\const A,\,\const\theta}{Konstanta}
\glsentm{o}{cvec}{\cvec a,\,\cvec\theta}{Konstanta vektor ili niz}
\glsentm{o}{cmat}{\cmat A,\,\cmat\Theta}{Konstanta matrica ili višedimensionalni niz}
\glsentm{o}{cset}{\cset A,\,\cset\Theta}{Kostanta skup}
\glsentm{o}{rvar}{\rvar a,\,\rvar A,\,\rvar\theta}{Slučajna varijabla}
\glsentm{o}{rvec}{\rvec a,\,\rvec\theta}{Slučajni vektor ili niz}
\glsentm{o}{rmat}{\rmat A,\,\rmat\Theta}{Slučajna matrica ili višedimensionalni niz}
\glsentm{o}{rset}{\rset A,\,\rset\Theta}{Slučajni skup ili multiskup}

% Konstante
\glsgroup{k}{Konstante}{}
\glsentm{k}{emptyset}{\cbr{}} 			{Prazni skup}
\glsentm{k}{nul}{\cvec 0}				{Nul-vektor}
\glsentm{k}{mati}{\cmat I,\,\cmat I_n}	{Matrica identiteta (s $n$ redaka i stupaca)}
\glsentm{k}{cset}{\N,\Z,\R,\C}			{Poznati skup}

% Skupovi i nizovi
\glsgroup{sn}{Skupovi i nizovi}{}
\glsentm{sn}{range}{a\bidot b} 			{Kraći zapis za $a,..,b$}
\glsentm{sn}{setrange}{\cbr{a\bidot b}} {Skup cijelih brojeva od $a$ do $b$}
\glsentm{sn}{setdefn}{\cbr{a_i\mid i=1\bidot n},\,\cbr{a_1\bidot a_n},\,\cbr{a_i}_{i=1\bidot n}}{Skup s $n$ elemenata}
\glsentm{sn}{setdefset}{\cbr{f(a)\mid P(a)},\, \cbr{f(a)}_{P(a)},\, \cbr{f(a)}_{a}}{Skup čiji su elementi definirani preko funkcije $f$ i predikata $P$ koji može biti implicitan ili neodređen}
\glsentm{sn}{ndarrdef}{\del{a_i}_{i}, \del{a_{i,j}}_{i,j}, \del{a_{i,j,k}}_{i,j,k}}{Višedimenzionalni niz s implicitnim ili neodređenim brojem elemenata}
\glsentm{sn}{intoo}{\intoo{a,b}}		{Otvoreni interval}
\glsentm{sn}{intcc}{\intcc{a,b}}		{Zatvoreni interval}

% Indeksiranje
\glsgroup{i}{Indeksiranje}
{Indeksi elemenata vektora ili višedimenzionalnih nizova se radi jednoznačnosti mogu pisati u indeksu oznake vektora u uglatim zagradama. Npr. ako je definiran vektor $\vec a=(a_1,.., a_n)^\transpose$, onda je njegov $i$-ti element $\vec a_{\sbr{i}}=a_i$.}
\glsentm{i}{vecelem}{\vec{a}_{\sbr{i}}} 
	{$i$-ti element vektora $\vec{a}$}
\glsentm{i}{subvec}{\vec{a}_{\sbr{i_1:i_2}}}{Vektor kojeg čine elementi $\vec{a}_{\sbr{i_1}}, \vec{a}_{\sbr{i_1+1}},.., \vec{a}_{\sbr{i_2}}$}
\glsentm{i}{subvecsk}{\vec{a}_{\sbr{(i_1\bidot i_n)}}}{Vektor kojeg čine elementi $\vec{a}_{\sbr{i_1}}, \vec{a}_{\sbr{i_2}},.., \vec{a}_{\sbr{i_n}}$}
\glsentm{i}{matelem}{\mat{A}_{\sbr{i,j}}}{Element $i,j$ matrice $\mat A$}
\glsentm{i}{matrow}{\mat{A}_{\sbr{i,:}}}{$i$-ti redak matrice $\mat A$}
\glsentm{i}{masubmat}{\mat{A}_{\sbr{:,i_1:i_2,j}}}{2-D odsječak 3-D niza $\mat A$}

% Operacije linearne algebre
\glsgroup{l}{Operacije linearne algebre}{}
\glsentm{l}{scalprod}{\braket{\vec a}{\vec b}} {Skalarni produkt, može biti i $\vec{a}^\transpose\vec{b}$}
\glsentm{l}{hadprod}{\vec a \odot \vec b}	{Umnožak po elementima; Hadamardov produkt}
\glsentm{l}{haddiv}{\vec a \oslash \vec b} {Dijeljenje po elementima}
\glsentm{l}{matmul}{\mat A \mat B}		{Matrično množenje}
\glsentm{l}{matinv}{\mat A^{-1}}			{Inverz matrice}
\glsentm{l}{transp}{\mat A^\transpose}	{Transponiranje}
\glsentm{l}{diag}{\diag\del{\vec{a}}}		{Dijagonalna matrica kojoj dijagonalu čini vektor $\vec a$}
\glsentm{l}{det}{\envert{\mat{A}}}		{Determinanta matrice $\mat A$}
\glsentm{l}{vecnorm}{\enVert{\vec a}_p}	{$\const L^p$-norma vektora $\vec a$}
\glsentm{l}{matnorm}{\enVert{\mat A}_p}	{Matrična $\const L^p$-norma matrice $\mat A$}
\glsentm{l}{frobnorm}{\enVert{\mat A}_\const{F}}	{Frobeniusova norma matrice $\mat A$}

% Diferencijalni račun
\glsgroup{d}{Diferencijalni račun}{}
\glsentm{d}{od}{\od{y}{x}} 				{Derivacija $y$ po $x$}
\glsentm{d}{pd}{\pd{y}{x}} 				{Parcijalna derivacija $y$ po $x$}
\glsentm{d}{grad}{\nabla_{\vec x}{y}} 	{Gradijent $y$ po $\vec x$}
\glsentm{d}{gradmat}{\nabla_{\mat X}{y}}	{Gradijent $y$ po $\mat X$}
\glsentm{d}{jacobi}{\pd{\vec y}{\vec x}} 	{Jakobijan $\mat J\in \R^{m\times n}$ za $\vec y\in\R^m$ i $\vec x\in\R^n$}
\glsentm{d}{int}{\int_{\set A}f(x)\dif x,\,\int_{x\in\set A}f(x)} {Određeni integral funkcije $f(x)$ po $x\in\set A$}
\glsentm{d}{int2}{\int f(x)\dif x,\,\int_x f(x)} {Određeni integral funkcije $f(x)$ po $x\in\set A$, gdje je $\set A$ poznat iz konteksta}

% Teorija vjerojatnosti i teorija informacije
\glsgroup{vi}{Teorija vjerojatnosti i teorija informacije}
{Svakoj slučajnoj varijabli $\rvar a$ jednoznačno je dodijeljena jedna razdioba $\p(\rvar a)$ (ili $\P(\rvar a)$) i funkcija gustoće vjerojatnosti (poopćena funkcija) $p_{\rvar a}(a)=\p(\rvar a=a)$. Funkcija gustoće vjerojatnosti se može napisati još na 2 načina. Najkraći zapis je $p(a)$, gdje se po slovu implicitno pretpostavlja slučajna varijabla označena istim slovom bez serifa. $\P(A)$ označava vjerojatnost događaja $A$, a $P_{\rvar a}(a)(\cdot)$ funkciju vjerojatnosti slučajne varijable $\rvar a$. Mogu se koristiti i druge oznake za funkciju vjerojatnosti ili funkciju gustoće vjerojatnosti.}
%TODO move to new page if too high
\glsentm{vi}{rvar}{\rvar a}{Slučajna varijabla}
\glsentm{vi}{rvarcond}{(\rvar a\mid\rvar b\sheq b),\,(\rvar a\mid b)}{Uvjetna slučajna varijabla}
\glsentm{vi}{rvarjoint}{(\rvar a,\rvar b)}{Združena slučajna varijabla}
\glsentm{vi}{distr}{\distrib{A},\,q}{Razdioba}
\glsentm{vi}{event}{\cbr{R(\rvar a)}} {Događaj koji uključuje slučajnu varijablu $\rvar a$, gdje je $R$ neki predikat}
\glsentm{vi}{prob}{\P(\cbr{R(\rvar a)}),\,\P(R(\rvar a))} {Vjerojatnost događaja $\cbr{R(\rvar a)}$}
\glsentm{vi}{distrrvar}{\P(\rvar a),\,\p(\rvar a)} {Razdioba slučajne varijable $\rvar a$; $\P$ ako je $\rvar a$ diskretna slučajna varijabla, a $\p$ ako nije ili ako se ne zna}
\glsentm{vi}{pmf}{P_{\rvar a}(a),\,P(a)} {Funkcija vjerojatnosti slučajne varijable $\rvar a$: $P_{\rvar a}(a)=\P(\rvar a\sheq a)$}
\glsentm{vi}{probdens}{\p(\rvar a\sheq a)} {Gustoća vjerojatnosti događaja $\rvar a=a$}
\glsentm{vi}{pdf}{p_{\rvar a}(a),\,p(a)} {Funkcija gustoće vjerojatosti slučajne varijable $\rvar a$}
\glsentm{vi}{pdfcond}{p_{\rvar a\mid b}(a),\,p(a\mid b)} {Gustoća vjerojatnosti za događaj $\cbr{\rvar a\sheq a\mid\rvar b\sheq b}$; $p_{\rvar a\mid b}(a)=\p(\rvar a\sheq a\mid\rvar b\sheq b)$}
\glsentm{vi}{pdfjoint}{p_{\rvar a,\rvar b}(a,b),\,p(a,b)} {Gustoća vjerojatnosti za događaj $\cbr{\rvar a\sheq a,\rvar b\sheq b}$; $p_{\rvar a,\rvar b}(a,b)=\p(\rvar a\sheq a,\rvar b\sheq b)$}
\glsentm{vi}{hasdistrib}{\rvar a \sim \distrib A,\, \p(\rvar a)=\distrib A} {\textit{Slučajna varijabla $\rvar a$ ima razdiobu $\distrib{A}$}}
\glsentm{vi}{hasdistribset}{\rvar a \sim \set A} 	{\textit{Slučajna varijabla $\rvar a$ ima takvu razdiobu da svi elementi (multi)skupa $\set A$ imaju vjerojatnost proporcionalnu višestrukosti ($\frac{1}{\envert{\set A}}$ za običan skup)}}
\glsentm{vi}{fromdistrib}{a\sim\distrib A} {\textit{$a$ se izvlači iz razidobe $\distrib{A}$}}
\glsentm{vi}{fromrvar}{a\sim \rvar a,\,a\sim \p(\rvar a)} {\textit{$a$ se izvlači iz razidobe $\p(\rvar a)$}}
\glsentm{vi}{E}{\E_{a\sim\rvar a} f(a),\,\E_{\rvar a}f(a)} {Očekivanje funkcije slučajne varijable $\rvar a$}
\glsentm{vi}{D}{\D_{a\sim\rvar a} f(a),\,\D_{\rvar a} f(a)} {Disperzija (varijanca) funkcije slučajne varijable $\rvar a$}
\glsentm{vi}{Cov}{\Cov(\rvar a,\rvar b)}		{Kovarijanca}
\glsentm{vi}{Gauss}{\mathcal{N}(\mu, \sigma^2)} {Normalna razdioba s učekivanjem $\mu$ i varijancom $\sigma^2$}
\glsentm{vi}{Ent}{\const{H}(\rvar a)}			{Shannonova entropija}
\glsentm{vi}{CEnt}{\const{H}(\rvar a, \rvar b)} {Unakrsna entropija}
\glsentm{vi}{Dkl}{\Dkl{\rvar a}{\rvar b}}		{Kullback-Leiblerova divergencija}

% Ostalo
\glsgroup{f}{Funkcije i operatori}{}
\glsentm{f}{func}{\funcdef{f}{\set A}{\set B}} {Funkcija s domenom $\set A$ i kodomenom $\set B$}
\glsentm{f}{funcdef}{x\mapsto g(x)} {Definicija funkcije; funkcija koja preslikava $x$ iz domene u $g(x)$ iz kodomene}
\glsentm{f}{dirac}{\dirac\del{\cdot}}	{Diracova delta razdioba; poopćena funkcija za koju vrijedi $\dirac(x)=0$ za $x\neq0$ i $\int_x\dirac(x)\dif x=1$}
\glsentm{f}{doublebracket}{\enbbracket{\cdot}} {Iversonova uglata zagrada; $\enbbracket{P}=\begin{cases} 1, & P \equiv \top \\ 0, & P \equiv \bot \end{cases}$}

